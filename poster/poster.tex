% Unofficial University of Cambridge Poster Template
% https://github.com/andiac/gemini-cam
% a fork of https://github.com/anishathalye/gemini
% also refer to https://github.com/k4rtik/uchicago-poster

\documentclass[final]{beamer}

% ====================
% Packages
% ====================

\usepackage[T1]{fontenc}
\usepackage{lmodern}
\usepackage[orientation=portrait,size=a0,scale=1.25]{beamerposter}
\usetheme{gemini}
\usecolortheme{nott}
\usepackage{graphicx}
\usepackage{booktabs}
\usepackage{tikz}
\usepackage{pgfplots}
\pgfplotsset{compat=1.14}
\usepackage{anyfontsize}

% for tables
\usepackage{multirow}
\usepackage{multicol}
\usepackage{transparent}

% for urls
\usepackage{hyperref}

% for subfigures
\usepackage{caption}
\usepackage{subcaption}

% for bibliography
\renewcommand{\refname}{Literature}
\usepackage[backend=biber]{biblatex}
\addbibresource{../presentation/references.bib}

% for inline code listings
\usepackage{listings}
\usepackage{xparse}
\usepackage{xcolor}
\definecolor{codeblue}{rgb}{0.0, 0.0, 1.0}
\definecolor{codegreen}{rgb}{0.0, 0.5, 0.0}
\definecolor{codegray}{rgb}{0.5, 0.5, 0.5}
\definecolor{codepurple}{rgb}{0.58, 0.0, 0.82}
\definecolor{backcolour}{rgb}{0.95, 0.95, 0.92}
\lstdefinestyle{py}{
    language=Python,
    %  backgroundcolor=\color{backcolour},
    commentstyle=\color{codegreen},
    keywordstyle=\color{codeblue},
    numberstyle=\tiny\color{codegray},
    stringstyle=\color{codepurple},
    basicstyle=\small\ttfamily,
    breakatwhitespace=false,
    breaklines=true,
    captionpos=b,
    keepspaces=true,
    numbers=left,
    numbersep=5pt,
    showspaces=false,
    showstringspaces=false,
    showtabs=false,
    tabsize=4,
    frame=0,
    rulecolor=\color{black},
    morekeywords={self, assert, yield, None, True, False}
}

% ====================
% Lengths
% ====================

% If you have N columns, choose \sepwidth and \colwidth such that
% (N+1)*\sepwidth + N*\colwidth = \paperwidth
\newlength{\sepwidth}
\newlength{\colwidth}
\setlength{\sepwidth}{0.025\paperwidth}
\setlength{\colwidth}{0.45\paperwidth}

\newcommand{\separatorcolumn}{\begin{column}{\sepwidth}\end{column}}

% ====================
% Title
% ====================

% Title page
\title{\Large Network Diffusion --- Framework to Simulate Spreading Processes in Complex Networks}
\author{\large
    \textbf{Micha{\l} Czuba} \inst{1},
    Mateusz Nurek \inst{1},
    Damian Serwata \inst{1},
    Yu-Xuan Qi \inst{2},
    Mingshan Jia \inst{2},
    Katarzyna Musial \inst{2},
    Rados{\l}aw Michalski \inst{1},
    Piotr Br{\'o}dka \inst{1}
}
\institute{
  \inst{1} Wroc{\l}aw University of Science and Technology\\
  \inst{2} University of Technology Sydney
}

% ====================
% Footer (optional)
% ====================

\footercontent{
  \href{https://networks.pwr.edu.pl/}{https://networks.pwr.edu.pl} \hfill
  NetSci, Canada, July 2023 \hfill
  \href{mailto:michal.czuba@pwr.edu.pl}{michal.czuba@pwr.edu.pl}}

% ====================
% Logo (optional)
% ====================

\logoright{\includegraphics[height=4cm]{logos/nsl-white.pdf}}
\logoleft{\includegraphics[height=4cm]{logos/wust_logo.pdf}}

% ====================
% Body
% ====================

\begin{document}

\begin{frame}[t, fragile]
\begin{columns}[t]

\separatorcolumn
\begin{column}{\colwidth}

\begin{block}{Introduction}
    Spreading phenomena are one of the issues considered by a network science. They can be obeserved
    in various areas like: dynamics of political opinions, marketing campaigns, spread of epidemics,
    etc. With the advancement of network science analytical approaches became insufficient for large
    graphs, prompting researchers to use computational methods. In recent years, the scope of network science has significantly expanded beyond static graphs to encompass more complex structures. The introduction of streaming, temporal, multilayer, and hypernetwork approaches has brought new possibilities and imposed additional requirements. Unfortunately, the pace of advancement is often
    too rapid for existing computational packages to keep up with the functionality updates...
\end{block}

\begin{alertblock}{Problem}
    There is a bunch of very good and robust tools that helps in sumulating diffusion processes in 
    networks, e.g. \lstinline[style=py]{ndlib} (which we love). \\
    \vspace{2em}
    However, if we consider...\\
    \vspace{1em}
    \hspace{5em}...more complex network models,... \\
    \vspace{1em}
    \hspace{5em}...spreading multiple processes at the same time... \\
    \vspace{1em}
    \hspace{14em}...\textbf{a gap among the available toolkits} emerges.
\end{alertblock}

\begin{block}{Our Contribution}
    In order to address the issue, we decided to redesign, polish, and share our internal 
    environment which we are using in the lab. Thus, in our recent work~\cite{czuba2024networkdiffusion},
    we presented \lstinline[style=py]{network-diffusion}. To start using the library just type in 
    your shell:
    \begin{center}
        \large
        \begin{verbatim}
            pip install network-diffusion
        \end{verbatim}
    \end{center}
    \vspace{-1em}
    Or scan this code to reach the docs with more examples:
    \begin{figure}
        \includegraphics[width=15cm]{../presentation/figures/qr_code.pdf}
    \end{figure}
\end{block}

\begin{block}{Key Features of \lstinline[style=py, basicstyle=\large\ttfamily]{network-diffusion}}
    \begin{itemize}
        \item \textbf{End-to-End Simulation Workflow}
        \item \textbf{Support for Temporal Network Models}
        \item \textbf{Support for Multilayer Network Models}
        \item \textbf{A Bunch of Predefined Spreading Models}
        \item \textbf{An Interface for Implementing Custom Spreading Models}
        \item \textbf{New Centrality Measures}
        \item \textbf{NetworkX Compatibility}
    \end{itemize}
\end{block}

% \begin{exampleblock}{Example --- Linear Threshold Model in Multilayer Networks}
\begin{exampleblock}{Example}
    \heading{Linear Threshold Model in Multilayer Networks}
    In a diffusion under Linear Threshold Model, each node:
    \begin{itemize}
        \item can fall in two states: \textit{active} and \textit{inactive},
        \item becomes \textit{active} if the fraction of its \textit{active} neighbors to all 
        neighbours exceeds certain threshold ($\mu$).
    \end{itemize}
    In case of multilayer networks --- actors are the subject of the process, while the 
    nodes are their auxiliary representation. Thus, we have to define how to aggregate impulses 
    from the layers. In this example we will consider $OR$ strategy (aka protocol) which says that
    the actor can be activated if any of nodes representing it gets activated.
\end{exampleblock}

\end{column}

\separatorcolumn
\begin{column}{\colwidth}

\begin{exampleblock}{Example}

\heading{Scenario to Simulate}
A Figure below shows the case we would like to model with \lstinline[style=py]{network-diffusion} 
--- MLTM with homogeneous threshold $\mu=0.3$ and protocol $OR$. The process will diffuse in a
network with eleven actors, starting from the agent six and ten.
\begin{figure}
    \centering
    \includegraphics[width=1\linewidth]{../presentation/figures/ltm_example_or.pdf}
\end{figure}

\heading{Modelling this case with \lstinline[style=py, basicstyle=\large\ttfamily]{network-diffusion}}

\begin{lstlisting}[style=py, basicstyle=\footnotesize\ttfamily]
import network_diffusion as nd

# get the graph - a medium for spreading
net = nd.mln.functions.get_toy_network_piotr()

# set actor 6 and 10 as seeds (we can use another heuristic here as well)
ac_6, ac_10 = net.get_actor(6), net.get_actor(10)
ranking_list = [ac_6, ac_10, *set(net.get_actors()).difference({ac_6, ac_10})]
seed_selector = nd.seeding.MockingActorSelector(ranking_list)
seed_quota = 100 * 2 / net.get_actors_num()

# define the model according to the given parameters
spreading_model = nd.models.MLTModel(
    seeding_budget=[100 - seed_quota, seed_quota],
    seed_selector=seed_selector,
    protocol="OR",
    mi_value=0.3,  
)

# perform the simulation that lasts four epochs
simulator = nd.Simulator(model=spreading_model, network=net)
logs = simulator.perform_propagation(n_epochs=3)

# obtain detailed logs for each actor in the form of JSON
raw_logs_json = logs.get_detailed_logs()

# or obtain aggregated logs for each of the network's layer
aggregated_logs_json = logs.get_aggragated_logs()

# or just save a summary of the experiment with all the experiment's details
logs.report(visualisation=True, path="my_experiment")
\end{lstlisting}

\heading{Outcomes}

The output logs can be used for further analysis and contain:
% \begin{itemize}
%     \item a description of the network,
%     \item a description of the propagation model,
%     \item a report of the spreading for all simulated phenomena,
%     \item a capture of states of each node during the simulation,
%     \item a brief visualisation of the propagation.
% \end{itemize}
% .. figure:: images/experiment_vis.png
%     :width: 600

\begin{figure}
    \centering
    \begin{subfigure}[b]{0.395\textwidth}
        \centering
        \fbox{\includegraphics[width=\textwidth]{figures/output.png}}
    \end{subfigure}
    \begin{subfigure}[b]{0.48\textwidth}
        \centering
        \fbox{ \includegraphics[width=\textwidth]{figures/visualisation.png}}
    \end{subfigure}
    % \caption{kkk}
\end{figure}

\end{exampleblock}

\begin{block}{References}
    \printbibliography
\end{block}

\vspace{30em}

\end{column}
\separatorcolumn

\end{columns}
\end{frame}

\end{document}
